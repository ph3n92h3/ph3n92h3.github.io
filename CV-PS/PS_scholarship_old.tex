% 区分不同奖学金,共有三处修改
\documentclass{ctexart}

\usepackage{enumitem}
\usepackage[a4paper, scale = 0.8]{geometry}
\usepackage{hyperref}
\usepackage{siunitx}

% \title{2022年度本科生国家奖学金个人申报材料}
% \title{2022年度河海大学“小米奖学金”个人申报材料}
% \title{“安徽邦泰优秀学生奖学金”个人申报材料}
\title{2023 年度“严恺奖学金”个人申报材料}
\author{河海大学\ 理学院\ 应用物理学\ 冯哲\ 2010020129}

\begin{document}
\maketitle

我是冯哲,来自河北省邯郸市,以 641 分的高考成绩进入河海大学,成为 2020 级应用物理学一班的一名学生。在三年来的大学生活中,我刻苦奋斗,在学校老师的关心和同学的帮助下,各方面都取得了优异的成绩:在思想上积极向党组织靠拢,努力成为了一名光荣的中共预备党员,时刻坚持以党员标准严格要求自己;在学习上,三年的刻苦学习使我取得并保持了专业第一的优秀成绩,顺利通过英语六级和计算机四级,同时还获得了多次获得高等数学比赛、数学建模比赛等重大赛事的奖项;工作方面,我曾担任班级心理委员,为班级建设贡献自己的力量,帮助班级获得了各种团体奖项;我还积极参加了各方面的社会实践,包括在家乡担任抗疫志愿者、主动提供献血六次、在九龙湖阅读中心担任志愿者……我认为自己思想进步、学习刻苦、工作认真、成绩优异,达到了
% 2022年度本科生国家奖学金
% 2022年度河海大学“小米奖学金”
% “安徽邦泰优秀学生奖学金”
2023 年度“严恺奖学金”的要求,现主动申报。个人情况的细节如下。

\section{思想政治}

为了提高自身的政治理论修养,我坚持学习马列主义、毛泽东思想、中国特色社会主义理论和习近平新时代中国特色社会主义思想,用科学的理论知识来武装我的头脑。经同学们的投票认可,我成为了入党积极分子,并在进一步加强政治理论学习和个人综合素质的提升之后,成为了一名光荣的 \textbf{中共预备党员},将会很快转正。坚持学习马列主义理论,不仅使我在多门政治理论课程中取得了优秀成绩,更让我逐渐领悟唯物辩证法的科学内涵,开始用全面深邃的眼光来观察世界和自身,获得了精神上的升华。

在日常生活中,我树立了正确的人生观和道德观。同时,我还时刻关注党和国家、世界和社会的发展形势与局势变化。在现实生活中和互联网上,我也勇于和破坏祖国统一、民族团结、人民幸福生活的行为作斗争。

我非常重视自己思想政治素养的提高,积极参加相关活动,并在其中获得了以下荣誉:

\begin{itemize}[noitemsep,nolistsep]
    \item “学四史铭初心·迎百年担使命”河海大学第二届“大学生讲思政课”公开课比赛 \hfill \textit{优秀奖}
    \item 河海大学“网上重走长征路”暨推动“四史”学习教育线上知识竞答活动 \hfill \textit{优秀个人\ $\times 2$}
    \item 理学院“读一本有意义的书,做一件有意义的事”活动  \hfill \textit{三等奖\ $\times 3$}
\end{itemize}

\section{学习与科研}

进入大学以来,我在学习上严格要求自己,取得了优异的成绩。

\subsection{课程学习}

从朱院长的“开学第一课”到现在,我在每堂课上的认真听讲,积极与老师互动,每门课程都取得了令人满意的成绩,\textbf{总绩点4.85,排名专业第一}。出于对数学、物理及其应用的浓厚兴趣,我报名了各类 \textbf{高等数学竞赛} 和 \textbf{数学建模比赛},获得了如下奖项:

\begin{itemize}[noitemsep,nolistsep]
    \item 江苏省高等学校第二十届高等数学竞赛\hfill \textit{本科一级 A 组二等奖}
    \item 2022年第八届全国大学生物理实验竞赛 \hfill \textit{二等奖}
    \item 二零二二年高教社杯全国大学生数学建模竞赛 \hfill \textit{本科组二等奖}
    \item 江苏省高等学校第十九届高等数学竞赛 \hfill \textit{本科一级 A 组一等奖}
    \item 美国大学生数学建模竞赛 \hfill \textit{Honorable Mention}
    \item 第十三届全国大学生数学竞赛 \hfill \textit{(非数学类)一等奖}
    \item 江苏省高等学校第十八届高等数学竞赛 \hfill \textit{本科一级 A 组一等奖}
\end{itemize}

\subsection{英语和计算机}

进入大学以后,我逐渐意识到英语对于未来学习的重要作用,在英语学习上苦下功夫,并在英语等级考试中取得了 \textbf{英语四级 483 分,英语六级 539 分} 的优秀成绩。另外,我从小对于国外的文学作品兴趣浓厚,因而对翻译工作也格外喜欢。于是我两次参加学校组织的翻译比赛并获奖:

\begin{itemize}[noitemsep,nolistsep]
    \item 河海大学2021年“Valuing water”翻译大赛 \hfill \textit{汉译英组二等奖}
    \item 河海大学2023年“Accelerating Change”翻译大赛 \hfill \textit{英译汉组三等奖}
\end{itemize}

作为一名新时代的大学生,充分掌握计算机与信息技术知识对于我们更好地利用现代化技术至关重要。因此我除了在学校开设的计算机类课程中认真学习之外,还不断扩充自己的计算机知识,通过了各种 \textbf{计算机等级考试}:

\begin{itemize}[noitemsep,nolistsep]
    \item 全国计算机等级考试 \hfill \textit{四级\ 数据库工程师}
    \item 全国计算机等级考试 \hfill \textit{三级\ 数据库技术}
    \item 江苏省高等学校计算机等级考试 \hfill \textit{三级\ 软件技术及应用}
    \item 全国计算机等级考试 \hfill \textit{二级\ C++ 语言程序设计}
    \item 江苏省高等学校计算机等级考试 \hfill \textit{二级\ Visual C++}
\end{itemize}

\subsection{科学研究}

我所学习的应用物理学是一门基础科学,在前沿的科学研究中有广阔的天地等待我们探索。我在大一时即开始主动了解物理学前沿的知识和进展。在大二上学期,我正式进入了邵智斌老师的课题组,展开了半导体器件与纳米光电子方向的研究。在大二下学期参与了 \textbf{两项大学生创新训练计划项目} 如下:

\begin{itemize}[noitemsep,nolistsep]
    \item 基于Fabry-P{\'e}rot多层膜/Si结构的小型化波长可分辨光电探测器 \hfill \textit{第一主持人,优秀结题}\begin{itemize}[noitemsep,nolistsep]
              \item 硅的吸收光谱对于光的波长没有选择性,这导致现有的光电探测器难以实现对光谱的分辨,使得光电探测器的应用场景受到局限。而 Fabry-P\'{e}rot 多层膜由于其高度的灵活性与强大的波长选择性能有望解决此问题,将 Fabry-P\'{e}rot 多层膜与硅基半导体相耦合,可以在进行光探测的同时进行波长的选择,从而实现波长分辨。
              \item 传统的大型、固定的光谱仪通常需要长光路和宽接收面,难以满足时效性、便携性、小型化的应用需求。 光电探测器基于电极层和单晶硅,光电特性基于半导体的内禀性质,不依赖于长光路和宽接收面,将其应用于光谱分辨则可解决传统光谱仪的尺寸局限性问题。
          \end{itemize}
    \item 激光刻蚀辅助硅微纳结构图案化制备研究 \hfill \textit{成员,优秀结题}\begin{itemize}[noitemsep,nolistsep]
              \item 激光具有单色性好、方向性好、高精度、可设计性高等特点,与其他微纳结构材料制备方法相比,激光加工具有设备简单、任意性高、参数容易调控等优点,研究激光刻蚀在制备硅微纳结构中的应用有重要意义。
              \item 光伏电池生产中,利用表面制绒技术制备硅微纳结构,提高面板的吸光率和光电转换效率。然而该技术也使得硅基光伏面板呈现单一深色,降低光伏面板的美观性。通过精确调控硅微纳结构的尺寸和位置,可以控制硅晶圆局部光学特性,有望实现图案化光伏面板的制备,推动可装饰太阳能产业的发展。
          \end{itemize}
\end{itemize}

出于个人对引力和天文的兴趣,我独立展开了对于引力理论、致密星体、宇宙学暴胀的研究,并有两篇 \textbf{论文} 在投,见 \url{https://arxiv.org/abs/2210.01574}、\url{https://arxiv.org/abs/2211.13233}。

\begin{itemize}[noitemsep,nolistsep]
    \item Charged anisotropic white dwarfs in $f\left(R, T\right)$ gravity, \textbf{Z. Feng}, arXiv:2210.01574[gr-qc] \begin{itemize}[noitemsep,nolistsep]
              \item 在 $f\left(R, T\right) = R + 2 \beta T$ 引力的背景下,其中 $R$ 是 Ricci 标量,$T$ 是能量动量张量的迹,研究了带电各向异性白矮星(WD)的平衡结构。推导了一般情况下的恒星方程,并对 Chandrasekhar 状态方程(EoS)和与能量密度成比例的电荷密度分布 $\rho_{ch} = \alpha \rho$ 找到了数值解。通过调整不同的参数,比较了不同条件下的解的特性。最重要的是,通过以各种方式超越 GR 中的平凡 WD,解可能表现出超 Chandrasekhar 行为。本文对 WD 结构的研究所得到的结果可能会对天文观测,如超亮 Ia 型超新星,起对照作用。
          \end{itemize}
    \item Slow-roll inflation in $f\left(R, T, R_{ab}T^{ab}\right)$ gravity, \textbf{Z. Feng}, arXiv:2211.13233[gr-qc] \begin{itemize}[noitemsep,nolistsep]
              \item 在 $f\left(R, T, R_{ab}T^{ab}\right)$ 引力理论的框架下,研究了宇宙膨胀的慢滚近似,其中 $T$ 是 能量动量张量 $T^{ab}$、$R$ 和 $R_{ab}$ 分别是 Ricci 标量和张量。 从空间平坦 FLRW 度规中的作用原理得到引力场的运动方程后,引入暴胀标量场作为物质,仅考虑最小曲率-暴胀耦合项,得到该理论的基本方程。值得注意的是,在采用慢滚近似后,推导出了与带 $RT$ 混合项的 $f(R, T)$ 引力相同的方程。对不同领域的几个感兴趣的势场进行了单独计算,计算了慢滚参数和 e-folding 数 $N$。最后,我们在忽略了度规微扰的情况下,分析了暴胀标量场在扰动下的行为。这项研究进一步完善了修改引力理论中的慢滚暴胀。
          \end{itemize}
\end{itemize}

\section{学生工作与社会实践}

在大学生活的第一年,我担任 \textbf{班级心理委员},积极关注同学们的心理健康,为心理健康知识的普及贡献自己的一份力量。之后,我仍然全力支持班委同学的工作,为班级的建设发光发热,帮助班集体、团支部获得了以下荣誉:

\begin{itemize}[noitemsep,nolistsep]
    \item 河海大学2022年度“五四红旗团支部”
    \item “百强千优”基层团支部培养对象评选 \hfill \textit{二等奖}
\end{itemize}

在课余时间和假期,我热心于社会实践和志愿服务。在 2020 $\sim$ 2022 的寒暑假,我投身于家乡的防疫工作,多次担任 \textbf{抗疫志愿者},并获得了相关服务证书。我六次参与 \textbf{无偿献血} 的活动,累计献血 \SI{1700}{\milli\liter}。另外,我还曾在江宁区九龙湖阅读中心担任志愿者。

在进行社会实践的过程中,我一方面关心社会,了解了许多法律知识;另一方面关注个体,学习了不少安全知识。利用这些额外的收获,我也在相关的比赛中获奖:

\begin{itemize}[noitemsep,nolistsep]
    \item 河海大学第三十二届校园科技节暨第十八届金水节第四届法律常识大赛 \hfill \textit{二等奖}
    \item 2020 年新生安全知识测试和竞赛 \hfill \textit{二等奖}
\end{itemize}

\section{生活作风}

在宿舍生活中,我和室友重视维护宿舍卫生,我们的宿舍环境受到了同学、宿管阿姨、老师的一致肯定,并连续获得“文明示范宿舍”的称号:

\begin{itemize}[noitemsep,nolistsep]
    \item 2020 年学生军训“内务优秀宿舍”
    \item 2020 - 2021 学年 “文明示范宿舍”
    \item 2021 - 2022 学年 “文明示范宿舍”
    \item 2022 - 2023 学年 “文明示范宿舍”
\end{itemize}

在日常生活中,我发扬艰苦朴素的校风校训,厉行勤俭节约,反对铺张浪费,保持着朴素节俭的生活。虽然学校离家很远,但我仍然不忘和家里常联系,经常和父母长辈交流。在与人交往时,我性格开朗,严以律己、宽以待人,并积极帮助他人,和老师同学建立了良好的关系。

综合我的以上表现,我获得了多项荣誉称号和奖学金:

\begin{itemize}[noitemsep,nolistsep]
    \item 2021-2022 学年河海大学“优秀学生标兵”
    \item 河海大学 2021-2022 学年学业优秀奖学金
    \item 河海大学 2021-2022 学年科技创新奖学金
    \item 河海大学 2021-2022 学年精神文明奖学金
    \item 河海大学 2020-2021 学年学业优秀奖学金
    \item 河海大学 2020-2021 学年科技创新奖学金
\end{itemize}

\bigskip \hrule \bigskip

% 以上是我个人情况的介绍。我认为自己已经达到了国家奖学金的要求,现提交申请。
% 以上是我个人情况的介绍。我认为自己已经达到了河海大学“小米奖学金”的要求,现提交申请。
% 以上是我个人情况的介绍。我认为自己已经达到了“安徽邦泰优秀学生奖学金”的要求,现提交申请。
以上是我个人情况的介绍。我认为自己已经达到了“严恺奖学金”的要求,现提交申请。
若我有幸获得,则更应该在学习和生活中以身作则,起到榜样带头作用,绝不玷污这一荣誉;若我遗憾未获得,也会戒骄戒躁、继续努力,继续向其他优秀的同学看齐,以更高的标准要求自己。

\end{document}