\documentclass[fontset=founder]{ctexart}

\usepackage[a4paper,scale=0.85]{geometry}

\usepackage[colorlinks=true]{hyperref}
\usepackage{physics}

\newcommand{\school}{}
\newcommand{\subject}{}

\title{个人陈述}
\author{河海大学\ 理学院\ 冯哲}
% \date{}

\begin{document}
\maketitle

在第三年还没有结束的时候就试图去对过去的大学生活做一个这样的总结,听上去有些前不着村后不着店,但是小到个人生老病死,大到人类社会历史总是尘土飞扬、兵荒马乱,趁这样一个夏令营报名的机会来回顾入学以来的匆忙也未尝不可。

经历了十五年的应试教育进入大学,我开始继续追逐我的理论物理之梦,在大一和大二上学期,我在四大力学的广阔原野上自由地奔跑。那段时间经常上在知乎上看到一些物理前辈指点江山激昂文字,时常感到自己初入道途、前路漫漫。偶尔,也会因为不知道自己未来能做什么感到迷茫和不知所措。依稀记得大一时的晚自习,我在教室里带着耳机听赵峥先生的广义相对论课——那应该是我第一次认真学习广义相对论。在大二上学期,我加入了邵智斌老师的课题组,开展了一些半导体领域的研究,阅读了大量半导体、光电探测器领域的文献,同时主持和参与各一个大学生创新训练项目(目前已经顺利结题)。那时候的我像一个原野上的游荡者,漫无目的地随机行走,尽管不经意间也有精美的宝石收获。

一切的转变发生在 2022 年 2 月 16 日,梁灿彬先生之噩耗传来。当时我正坐在离家返校的火车上,看到消息我如同被闪电击中,从中学时代对电磁学的印象,到网络上对这位老师的甚高评价,再到不久前无心之间看过的几次讲课视频在我的脑海里一闪而过,我当即觉得那是一个不平凡的时刻,我应该做出真正的改变,应该坚定地朝着一个方向努力。我从一个在海边捡贝壳,时不时看一看大海尽头地平线的小孩,逐渐变成了一个驾扁舟乘长风破万里浪的前进者。我在心中暗暗告诉自己,一定要学好引力理论,坚定不移地追求自己的梦想。于是大二下学期开始我专攻高能理论,先是广义相对论,然后是量子场论。在 2022 年的暑假,看了一天的书之后散步去批发店买一块雪糕,是我当时最快乐的事情。

实际上,我在 2022 年暑假继续学习《微分几何入门与广义相对论》的时候就开始每天追踪 arXiv 的动态,并发现修改引力是一个有趣又非常新手友好的方向。那段时间我看了很多修改引力的文章,既有最新的也有原始的,既有宇宙学的也有致密星的。后来我看到了一系列关于修改引力和白矮星、中子星、夸克星的文章,这个领域已经发展得较为完善了,但是我还是找到了一个可以做的题目。最终我完成了它,即 \textit{$f \qty(R, T)$ 引力下的各向异性带电白矮星} \footnote{Charged anisotropic white dwarfs in $f \qty(R, T)$ gravity, Zhe Feng, \href{https://arxiv.org/abs/2210.01574}{arXiv:2210.01574[gr-qc]}}. 当时我认为经典引力主要有四个大方向:黑洞、宇宙学、引力波、天体,刚刚完成的这个算是天体,于是我试图寻找其他三个方向的题目。
\begin{quote}
    \textit{The career of a young theoretical physicist consists of treating the harmonic oscillator in ever-increasing levels of abstraction.} by Sidney Coleman
\end{quote}
斯言不谬,我的确研究了一个谐振子的行为——只不过这个谐振子是早期宇宙的暴胀子,即 \textit{$f \qty(R, T, R_{ab}T^{ab})$ 引力下的满滚暴胀} \footnote{Slow-roll inflation in
$f \qty(R, T, R_{ab}T^{ab})$ gravity, Zhe Feng, \href{https://arxiv.org/abs/2211.13233}{arXiv:2211.13233[gr-qc]}}.

To tell the truth, 这样的文章我可以继续写下去,但是我在第二篇完成之后就停止了。一方面实际上写出这两篇文章的那段时间我并不快乐,除了繁杂的计算,排版的琐事和学学校安排的课程等使我难以再同时兼顾;另一方面,与其他论文的对比和接二连三的拒稿都明显说明我的水平离一个合格的职业理论家十万八千里。

于是我开始了继续的学习——某种程度上也可以说是上过战场之后的训练。我开始专攻宇宙学,尤其是宇宙学微扰、暴胀等话题,在我眼里他们精致而美丽。另外,量子场论从另一个角度吸引了我:
\begin{quote}
    \textit{If we consider protons and neutrons as elementary particles, we would have three kinds of elementary particles [p,n,e]... This number may seem large but, from that point of view, two is already a large number.} by Paul Dirac
\end{quote}
仰观宇宙之大,俯察品类之盛,进而羡长江之无穷,可是如何做到识盈虚之有数?量子场论能告诉我们世界是由什么组成的,而它涉及的物理思想和数学技术也十分美妙。我在学习引力理论的同时也开始通过各种方式学习量子场论、群论,包括读书、看视频、与他人讨论等等,直到现在。

\bigskip \hrule \bigskip

量子力学老师鼓励我们说:
\begin{quote}
    \textit{我的理论水平马马虎虎,教课大体上不会有太多错误。但在前行中,时常也会感觉自己驽钝和无知。在基础科学领域,要想走的更远,不仅仅是凭借一点聪明才智。若是你天资聪颖,万里挑一,其实在我看来不过尔尔,因为物理学中天才辈出。若是你资质普通,没有背景,也不必安自菲薄,大侠郭靖也非天选之人。青,取之于蓝,而青于蓝;冰,水为之,而寒于水。我很喜欢我们专业里面有个性有思想的同学,无论你们以后是否从事物理,希望你们不忘初心,方得始终。}
\end{quote}
物理专业的人时常抛出那个古老的历史自嘲:
\begin{quote}
    \textit{Ludwig Boltzmann, who spent much of his life studying statistical mechanics, died in 1906, by his own hand. Paul Ehrenfest, carrying on the work, died similarly in 1933. Now it is our turn to study statistical mechanics.} by David Goodstein
\end{quote}
但是——尽管说这种话对于一个二十岁的年轻人来说有些许异样——整个人类社会的绝大多数活动看上去都是机械、繁复、无聊的。然而我还是想在这里引用 Steven Weinberg 箴言:
\begin{quote}
    \textit{The effort to understand the universe is one of the very few things that lifts human life a little above the level of farce, and gives it some of the grace of tragedy.}
\end{quote}
尝试理解宇宙不会给我带来物质上的享受,也不会给我带来声誉的提高,但是这是我能想到为数不多的一种值得我们毕生追求的事业。

\begin{quote}
    \textit{Wer mit Ungeheuern kämpft, mag zusehn, dass er nicht dabei zum Ungeheuer wird. Und wenn du lange in einen Abgrund blickst, blickt der Abgrund auch in dich hinein.} by Friedrich Wilhelm Nietzsche
\end{quote}
可能听上去有点天真——但是,在我凝视物理的时候,物理会不会低下头来看我一眼?在我为物理奉献自己的热情与青春的时候,物理会不会眷顾我?

我自知没有什么资格去做主观的评价,但是从客观上来说,\school 的物理学水平非常高,足以吸引我在这里继续我的学习与研究生涯。我对\subject 尤其感兴趣,希望能到\school 继续求学。

这篇个人陈述到此为止了,然而我还是会不断地修改它,在我的生命中这种东西好像带着点什么宿命的意味。我希望我能渐渐的忘掉它,就像后世的人总会渐渐忘掉我;但我又希望我永远会修改它,就像任何伟大的东西所经受的那样。

\end{document}