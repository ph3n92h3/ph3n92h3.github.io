\documentclass[12pt]{ctexart}

\pagestyle{plain}
\usepackage[a4paper, scale = 0.75]{geometry}
\usepackage{siunitx}

\title{2023 年度“钱正英助学金”个人申请材料}
\author{河海大学\ 理学院\ $\cdot$ 冯哲}

\begin{document}
\maketitle

冯哲,男,汉族,中共预备党员,理学院 2020 级应用物理学专业本科生,绩点 4.85,专业排名 1/56,曾获河海大学优秀学生标兵等荣誉称号,理学院 2021-2022 学年 “李立聪奖学金”、河海大学学业优秀奖学金、河海大学科技创新奖学金、河海大学精神文明奖学金等。

作为一名预备党员,冯哲政治立场坚定,热爱祖国,拥护中国共产党的领导,认真学习马克思列宁主义、毛泽东思想、中国特色社会主义、习近平新时代中国特色社会主义思想,自觉遵守国家各项法律法规、党规党纪和学校学院的规章制度,时刻坚持以党员标准严格要求自己,增强“四个意识”、坚定“四个自信”,不断提高自己的思想政治觉悟和政治理论水平,树立了正确的人生观和价值观,敬爱师长,团结同学,具备良好的道德品质。该生积极参加思想教育活动,在“学四史铭初心·迎百年担使命”河海大学第二届“大学生讲思政课”公开课比赛中获优秀奖,两次参与河海大学“网上重走长征路”暨推动“四史”学习教育线上知识竞答活动获优秀个人称号,三次参加理学院“读一本有意义的书,做一件有意义的事”活动获三等奖。

该生学习刻苦认真,踔厉奋发,数理基础扎实,学习成绩优异,成绩排名专业第一;
英语水平良好,在英语六级考试中取得 539 分的高分,参与河海大学2021年“Valuing water”翻译大赛获汉译英组二等奖,参与河海大学2023年“Accelerating Change”翻译大赛获英译汉组三等奖;
计算机水平优秀,先后取得了全国计算机等级考试 C++ 语言程序设计二级证书、数据库技术三级证书、数据库工程师四级证书,江苏省高等学校计算机等级考试 Visual C++ 二级证书、软件技术及应用三级证书。
该生具有较强的学习能力、分析问题能力和科技创新能力,
积极参加高等数学竞赛、大学生数学建模竞赛、大学生物理实验竞赛等重大赛事并屡创佳绩,获二零二二年高教社杯全国大学生数学建模竞赛本科组二等奖、2022 年第八届全国大学生物理实验竞赛二等奖、2022 年美国大学生数学建模竞赛 H 奖、第十三届全国大学生数学竞赛(非数学类)一等奖、江苏省高等学校第二十届高等数学竞赛本科一级 A 组二等奖、江苏省高等学校第十九届高等数学竞赛本科一级 A 组一等奖、江苏省高等学校第十八届高等数学竞赛本科一级 A 组一等奖等竞赛奖项,为学院和学校增光添彩。
在我校邵智斌教授的半导体与纳米光电子实验室学习工作,主持负责“基于 Fabry-P{\'e}rot 多层膜 / Si 结构的小型化波长可分辨光电探测器”校级创新训练项目并获优秀结题,基本满足了时效性、便携性、小型化的光谱仪应用需求;参与“激光刻蚀辅助硅微纳结构图案化制备研究”校级创新训练项目并获优秀结题,提出通过精确调控硅微纳结构的尺寸和位置,可以控制硅晶圆局部光学特性,有望实现图案化光伏面板的制备,推动可装饰太阳能产业的发展。
通过文献调研确定方向,开展了引力理论与致密星体方向的研究并以第一作者撰写了 \textit{Charged anisotropic white dwarfs in $f\left(R, T\right)$ gravity},研究了带电各向异性白矮星的平衡结构,目前已获多次引用;开展了引力理论与宇宙学方向的研究并以第一作者撰写了 \textit{Slow-roll inflation in $f\left(R, T, R_{ab}T^{ab}\right)$ gravity},研究了宇宙膨胀的慢滚近似,具有一定的科研水平和较高的科研潜力。

该生在担任心理委员期间,积极配合学校心理健康中心的工作,关心同学的心理健康,宣传普及心理健康知识,取得了相关证书。
为班级、团支部建设贡献力量,积极协助班级工作,帮助班级取得了河海大学 2022 年度“五四红旗团支部”、“百强千优”基层团支部培养对象评选二等奖等荣誉。
该生积极参加学院班级组织的集体活动和社会实践,包括在家乡担任抗疫志愿者、多次主动无偿献血(累计献血 \SI{1700}{\milli\liter})、在家乡的小学协助教学活动、在江宁区九龙湖阅读中心担任志愿者等。
参加社会实践的过程中掌握了更多的课外知识,参加河海大学第三十二届校园科技节暨第十八届金水节第四届法律常识大赛获二等奖,参加 2020 年新生安全知识测试和竞赛获二等奖,德智体美劳全面发展。

在宿舍生活中,冯哲同学重视宿舍卫生环境,受到了同学、宿管阿姨和老师的一致肯定,与宿舍同学一起获得2020 年学生军训“内务优秀宿舍”、2020-2021 学年 “文明示范宿舍”、2021-2022 学年 “文明示范宿舍”、2022-2023 学年 “文明示范宿舍”等荣誉称号。在日常生活中,发扬艰苦朴素的校风校训,厉行勤俭节约,反对铺张浪费,保持朴素节俭的生活习惯。在艰苦的条件下,体谅父母,关心家人,弘扬孝道传统美德。

以上是申请人冯哲的个人情况介绍。该生政治素质过硬,学习成绩优异,综合素质良好,在思想、学习、生活、社会实践等各方面均起到了榜样带头作用,考虑到该生家庭条件较为困难,故申请钱正英助学金。若有幸获得,则更应该在学习和生活中以身作则,起到榜样带头作用;若没有获得,也会戒骄戒躁、继续努力,继续向其他优秀的同学看齐,以更高的标准要求自己。

\end{document}