\documentclass[12pt]{article}

\usepackage{xeCJK}
\usepackage{enumitem}
\usepackage[scale=0.9]{geometry}
\usepackage[colorlinks=true]{hyperref}

\renewcommand*{\section}[1]{
    ~\\ \noindent \textbf{#1} \medskip \hrule \medskip
}

\begin{document}\pagestyle{empty}

% title

\begin{center}
    \LARGE{\textbf{冯\ 哲}}
\end{center}

\begin{center}
    \href{mailto:2010020129@hhu.edu.cn}{2010020129@hhu.edu.cn} $\diamond$ \href{https://ph3n92h3.github.io}{https://ph3n92h3.github.io}
\end{center}

% Education

\section{教育经历}

\textbf{中国科学院理论物理研究所} \hfill 2024.09 起

\textbf{河海大学理学院} \hfill 2020.09 至今

\smallskip \quad \textit{理学学士 \hfill GPA: 4.85/5.00, 排名 1/56}

\begin{itemize}[noitemsep,nolistsep]
    \item 力学: 92, 热学: 95, 电磁学: 94, 光学: 91
    \item 数学物理方法: 99
    \item 理论力学: 93,电动力学: 97,热力学与统计物理: 97,量子力学: 99,高等量子力学:93
    \item 固体物理学: 94,半导体物理学: 95
\end{itemize}

% Research Interest

\section{研究兴趣}

\textbf{主要兴趣}
\begin{itemize}[noitemsep,nolistsep]
    \item \textbf{广义相对论和量子宇宙学}\ 暴胀,宇宙学,黑洞,致密星体
\end{itemize}

\textbf{其他兴趣}
\begin{itemize}[noitemsep,nolistsep]
    \item \textbf{高能物理 - 理论}\ 超对称场论,共形场论,散射振幅
    \item \textbf{广义相对论和量子宇宙学}\ 宇宙早期热历史,暗物质,弯曲时空中的量子场论
\end{itemize}

% Research Experience

\section{研究经历}

\textbf{广义相对论和量子宇宙学} \hfill 2022.02 至今

\smallskip \quad \textit{修改引力与天体物理}

\textbf{Z. Feng}, \textit{Charged anisotropic white dwarfs in $f\left({R}, {T}\right)$ gravity}, \href{https://arxiv.org/abs/2210.01574}{arxiv:2210.01574[gr-qc]}

\begin{itemize}[noitemsep,nolistsep]
    \item 在 $f\left(R, T\right) = R + 2 \beta T$ 引力的背景下,其中 $R$ 是 Ricci 标量,$T$ 是能量动量张量的迹,研究了带电各向异性白矮星(WD)的平衡结构。推导了一般情况下的恒星方程,并对 Chandrasekhar 状态方程(EoS)和与能量密度成比例的电荷密度分布 $\rho_{ch} = \alpha \rho$ 找到了数值解。通过调整不同的参数,比较了不同条件下的解的特性。最重要的是,通过以各种方式超越 GR 中的平凡 WD,解可能表现出超 Chandrasekhar 行为。本文对 WD 结构的研究所得到的结果可能会对天文观测,如超亮 Ia 型超新星,起对照作用。
\end{itemize}

\smallskip \quad \textit{修改引力与宇宙学}

\textbf{Z. Feng}, \textit{Slow-roll inflation in $f\left(R, T, R_{ab}T^{ab}\right)$ gravity}, \href{https://arxiv.org/abs/2211.13233}{arxiv:2211.13233[gr-qc]}

\begin{itemize}[noitemsep,nolistsep]
    \item 在 $f\left(R, T, R_{ab}T^{ab}\right)$ 引力理论的框架下,研究了宇宙膨胀的慢滚近似,其中 $T$ 是 能量动量张量 $T^{ab}$、$R$ 和 $R_{ab}$ 分别是 Ricci 标量和张量。 从空间平坦 FLRW 度规中的作用原理得到引力场的运动方程后,引入暴胀标量场作为物质,仅考虑最小曲率-暴胀耦合项,得到该理论的基本方程。值得注意的是,在采用慢滚近似后,推导出了与带 $RT$ 混合项的 $f(R, T)$ 引力相同的方程。对不同领域的几个感兴趣的势场进行了单独计算,计算了慢滚参数和 e-folding 数 $N$。最后,我们在忽略了度规微扰的情况下,分析了暴胀标量场在扰动下的行为。这项研究进一步完善了修改引力理论中的慢滚暴胀。
\end{itemize}

\medskip \textbf{凝聚态 - 材料科学} \hfill 2021.09 $\sim$ 2023.08

\smallskip \quad \textit{基于 Fabry-P\'{e}rot 多层膜 / Si 结构的小型化波长可分辨光电探测器 \hfill 邵智斌\ 教授(河海大学)}

\begin{itemize}[noitemsep,nolistsep]
    \item 河海大学大学生创新训练项目 \textit{优秀结题}
    \item 硅的吸收光谱对于光的波长没有选择性,这导致现有的光电探测器难以实现对光谱的分辨,使得光电探测器的应用场景受到局限。而 Fabry-P\'{e}rot 多层膜由于其高度的灵活性与强大的波长选择性能有望解决此问题,将 Fabry-P\'{e}rot 多层膜与硅基半导体相耦合,可以在进行光探测的同时进行波长的选择,从而实现波长分辨。
    \item 传统的大型、固定的光谱仪通常需要长光路和宽接收面,难以满足时效性、便携性、小型化的应用需求。 光电探测器基于电极层和单晶硅,光电特性基于半导体的内禀性质,不依赖于长光路和宽接收面,将其应用于光谱分辨则可解决传统光谱仪的尺寸局限性问题。
\end{itemize}

\smallskip \quad \textit{激光刻蚀辅助硅微纳结构图案化制备研究 \hfill 邵智斌\ 教授(河海大学)}

\begin{itemize}[noitemsep,nolistsep]
    \item 河海大学大学生创新训练项目 \textit{优秀结题}
    \item 激光具有单色性好、方向性好、高精度、可设计性高等特点,与其他微纳结构材料制备方法相比,激光加工具有设备简单、任意性高、参数容易调控等优点,研究激光刻蚀在制备硅微纳结构中的应用有重要意义。
    \item 光伏电池生产中,利用表面制绒技术制备硅微纳结构,提高面板的吸光率和光电转换效率。然而该技术也使得硅基光伏面板呈现单一深色,降低光伏面板的美观性。通过精确调控硅微纳结构的尺寸和位置,可以控制硅晶圆局部光学特性,有望实现图案化光伏面板的制备,推动可装饰太阳能产业的发展。
\end{itemize}

% Academic Conference

\section{学术会议}

\begin{itemize}[noitemsep,nolistsep]
    \item 第三届“弦论,场论及全息理论”前沿暑期研讨会

    \hfill \textit{2023.08 - 中国,南京 - 东南大学丘成桐中心}
    \item 2023 年理论物理前沿讲习班——精密测量与引力性质检测

    \hfill \textit{2023.08 - 中国,扬州 - 扬州大学引力与宇宙学研究中心}
\end{itemize}

% Professional Skills

\section{专业技能}

\textbf{语言}
\begin{itemize}[noitemsep,nolistsep]
    \item 全国大学英语六级考试: 539
\end{itemize}

\textbf{程序设计}
\begin{itemize}[noitemsep,nolistsep]
    \item Mathematica(包括 \href{http://xact.es/index.html}{xAct}, \href{https://feyncalc.github.io/}{FeynCalc}, \href{https://arxiv.org/abs/1901.07808}{FIRE6}, \href{https://rulebasedintegration.org/}{Rubi}), Python
    \item GNU/Linux(日常使用 \href{https://archlinux.org/}{Arch Linux})
    \item \LaTeX
    \item 相关证书:\begin{itemize}[noitemsep,nolistsep]
              \item 全国计算机等级考试四级\ \textit{数据库工程师}
              \item 全国计算机等级考试三级\ \textit{数据库技术}
              \item 江苏省高等学校计算机三级\ \textit{软件技术及应用}
              \item 全国计算机等级考试二级\ \textit{C++ 语言程序设计}
              \item 江苏省高等学校计算机二级\ \textit{Visual C++}
          \end{itemize}
\end{itemize}

% Honors \& Awards

\section{荣誉奖项(部分)}

\begin{enumerate}[noitemsep,nolistsep]
    \item 2021-2022 学年河海大学“优秀学生标兵” \hfill \textit{2022.11}
    \item 河海大学 2021-2022 学年学业优秀奖学金 \hfill \textit{2022.11}
    \item 河海大学 2021-2022 学年科技创新奖学金 \hfill \textit{2022.11}
    \item 河海大学 2021-2022 学年精神文明奖学金 \hfill \textit{2022.11}
    \item 河海大学 2020-2021 学年学业优秀奖学金 \hfill \textit{2021.11}
    \item 河海大学 2020-2021 学年科技创新奖学金 \hfill \textit{2021.11}
          \\
    \item 江苏省高等学校第二十届高等数学竞赛本科一级A组\ \textit{二等奖} \hfill \textit{2023.06}
    \item 2022年第八届全国大学生物理实验竞赛\ \textit{二等奖} \hfill \textit{2022.12}
    \item 二零二二年高教社杯全国大学生数学建模竞赛本科组\ \textit{二等奖} \hfill \textit{2022.11}
    \item 江苏省高等学校第十九届高等数学竞赛本科一级A组\ \textit{一等奖} \hfill \textit{2022.11}
    \item 美国大学生数学建模竞赛\textit{Honorable Mention} \hfill \textit{2022}
    \item 第十三届全国大学生数学竞赛(非数学类)\ \textit{一等奖} \hfill \textit{2021.12}
    \item 江苏省高等学校第十八届高等数学竞赛本科一级A组\ \textit{一等奖} \hfill \textit{2021.06}
\end{enumerate}

% Volunteer Experience

\section{志愿服务}

\begin{itemize}[noitemsep,nolistsep]
    \item 累计献血六次,共 1700 mL \hfill \textit{2020 $\sim$ 2023}
    \item 河北省邯郸市临漳县新型冠状病毒疫情防控\ \textit{志愿者} \hfill \textit{2020 $\sim$ 2022}
    \item 江苏省南京市江宁区九龙湖阅读中心\ \textit{志愿者} \hfill \textit{2020}
\end{itemize}

\hfill {\tiny \today 更新}
\end{document}
